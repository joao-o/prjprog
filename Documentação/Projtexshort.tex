%\documentclass[a4paper,12pt]{article}
\documentclass{scrartcl}
\usepackage[latin1,utf8]{inputenc}       % Tipos de caracteres
\usepackage[portuges]{babel}             % Português
\usepackage[a4paper,portrait]{geometry}  % Tipo de papel
\usepackage{amsmath}                     % Extensões da American Mathematical Society
\usepackage{multicol}                    % Para tratar colunas multiplas
\usepackage{makeidx}                     % Para fazer índices
\usepackage{color}                       % Para tratamento da cor
\usepackage{fancyhdr}                    % Para cabeçalhos
\usepackage{url}                         % Para tratar endereços 'url
\usepackage{epstopdf}
\usepackage[pdftex]{graphicx}
\usepackage{graphics}
\usepackage{fancyhdr}
\usepackage{cleveref}
\usepackage{hyperref}

\oddsidemargin = 20pt                    % Margem do lado esquerdo: 31pt
\topmargin = 15pt                        % Margem superior: 20pt  
\headheight = 12pt                       % Tamanho do 'header': 12pt 
\headsep = 25pt                          % Espaço entre o 'header' e o texto: 25pt
\textheight = 592pt                      % Altura do texto: 592pt
\textwidth = 390pt                       % Largura do texto: 390pt
\marginparsep = 5pt                     % Espaço entre margem esquerda e o texto: 10pt
\marginparwidth = 20pt                   % Margem esquerda: 35pt
\footskip = 15pt                         % Espaço entre o texto e o 'footer': 30pt

\hyphenation{asso-ciada}
\input epsf

\makeindex

\begin{document}

\title{\bf Breve Introdução ao Projecto "Galileo"}
\subtitle{Um Simulador de Luneta Terrestre}
\author{João Oliveira e Tomás Reis \\ 
MEFT 1º Ano 1º Semestre \\ 
Instituto Superior Técnico  \\
Universidade de Lisboa\\}
\date{12 de Janeiro de 2013}
\maketitle

\section{Óptica da Luneta Terrestre}
A Luneta Terrestre foi desenhada e elaborada por Galileu. Trata-se de uma combinação bastante simples de uma lente convergente e uma lente divergente, em que a objectiva, a lente convergente, tem maior distância focal que a ocular, a lente divergente, estando a ocular colocada a uma distância da objectiva igual à diferença entre as distâncias focais.
Esta combinação é utilizada para ampliar objectos a uma distância tal que possa ser tida como infinita, obtendo uma imagem não invertida. 
\par
 
\section{Utilização do Programa}

Para começar a usar o programa basta, na linha de comandos, fazer {\bf make} na pasta disponibilizada para compilar os ficheiros e {\bf ./galileo} para o executar.

\subsection{Funcionalidades Básicas}
O objectivo do programa é permitir ao utilizador simular um sistema óptico que pode formar uma luneta, uma lente convergente e uma lente divergente, e as ferramentas para o utilizador o alterar.

\subsubsection{Posição das Lentes}
O programa permite ao utilizador alterar a posição das lentes de duas formas distintas: através do rato ou através das barras horizontais no primeiro separador. Para alterar a posição das lentes do rato basta arrastar uma das lentes com o rato. Caso as duas lentes estejam sobrepostas, o programa dará prioridade à lente convergente. Para alterar a posição das lentes com as barras de ajuste basta seleccionar o separador {\bf Posição das Lentes} e alterar a posição de cada lente na barra respectiva.

\subsubsection{Distância Focal das Lentes}

A distância focal de cada lente pode ser alterada de forma semelhante. No separador {\bf Distâncias Focais} existe uma barra de ajuste para cada distâncial focal, permitindo valores até 300.
A distância focal também pode ser alterada com o rato. Junto a cada lente existe um círculo. Este círculo pode ser arrastado com o rato, aproximando-o o afastando-o da lente. Aproximando este ponto da lente diminui a distância focal e afastando aumenta.
\par 

\subsubsection{Ângulo de Incidência}

Como o objecto observado pela luneta terrestre está muito distante, os raios provenientes são paralelos. Como tal, o parâmetro relevante sobre os raios que são recebidos pela luneta é o ângulo de incidência. Para alterar o ângulo de incidência basta utilizar a primeira barra de ajuste no separador {\bf Angulo/Escala}.

\subsection{Outras Opções do Programa}

\subsubsection{Escala}

No separador {\bf Angulo/Escala} é possível alterar a escala na segunda barra de ajuste. Alterar a escala reflecte uma multiplicação de todas as distâncias por um factor, ou seja funciona de forma semelhante a um zoom. Alterar a escala desliga a opção {\bf Fixar Distâncias}.
\par

\subsubsection{Raios Virtuais}

Na caixa {\bf Opções} está disponível a opção {\bf Ver Raios Virtuais}. Entende-se por raios virtuais todas as linhas que reflectem prolongamentos de raios luminosos. Quando esta opção se encontra ligada, estes raios têm uma cor diferente e são desenhadas a tracejados. Quando desligada, não são visíveis. Também torna visíveis/invisíveis as imagens virtuais.
\par

\subsubsection{Fixar Distâncias}

Na caixa {\bf Opções} está disponível a opção {\bf Fixar Distâncias}. Enquanto esta opção estiver ligada a distância entre as lentes será conservada quando uma das lentes é alterada. Isto limitará a alteração das lentes, de forma a que nunca seja possível arrastar uma lente para fora da área de desenho.
\par

\subsubsection{Recomeçar}

O botão {\bf Recomeçar} altera as todas definições ajustáveis pelo utilizador aos valores iniciais. Isto inclui todas as barras de ajuste e butões.
\par

\subsubsection{Criar Luneta}

O botão {\bf Criar Luneta} altera as posições das lentes de forma a que formem uma luneta terrestre. Para tal, a distância focal da lente convergente deve ser maior que a distância focal. É recomendado que a diferença entre distâncias focais seja grande para que se veja bem a luneta.
\par

\subsubsection{Cores}

O botão {\bf Cores} abre um menu que permite ajustar as cores dos objectos desenhados. Isto  inclui as lentes (no modo "esquemáticas), os raios reais e virtuais e os objectos. Neste menu o botão {\bf Restaurar Cor} reverte a cor seleccionada para a cor predefinida e o botão {\bf Cores Predefinidas} restaura todas as cores.
\par

\subsubsection{Bloqueado/Desbloqueado}

O botão {\bf Bloqueado/Desbloqueado} é semelhante ao botão {\bf Criar Luneta}, excepto que, enquanto estiver activo ("Bloqueado"), o programa força a existência de uma luneta. Isto é, por um lado, a posição das lentes não pode ser alterada directamente, e, por outro, caso as distâncias focais sejam alteradas a posição das lentes é automaticamente ajustada para formar uma luneta. Este botão bloqueia o botão {\bf Recomeçar}.

\subsubsection{Tipo de Lentes} 

A caixa {\bf Tipo de Lentes} apresenta duas opções para o desenho das lentes. A opção {\bf Esquemáticas} desenha as lentes como rectas encabeçadas por triângulos, como é padrão em esquemas de sistemas ópticos. A opção {\bf Desenhadas} desenha de uma forma ilustrativa as lentes, com um perfil de lente esférica.
\par

\par
\end{document}

-----------> footnotes
